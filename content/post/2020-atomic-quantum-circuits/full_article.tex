%% ****** Start of file apstemplate.tex ****** %
%%
%%
%%   This file is part of the APS files in the REVTeX 4 distribution.
%%   Version 4.1r of REVTeX, August 2010
%%
%%
%%   Copyright (c) 2001, 2009, 2010 The American Physical Society.
%%
%%   See the REVTeX 4 README file for restrictions and more information.
%%
%
% This is a template for producing manuscripts for use with REVTEX 4.0
% Copy this file to another name and then work on that file.
% That way, you always have this original template file to use.
%
% Group addresses by affiliation; use superscriptaddress for long
% author lists, or if there are many overlapping affiliations.
% For Phys. Rev. appearance, change preprint to twocolumn.
% Choose pra, prb, prc, prd, pre, prl, prstab, prstper, or rmp for journal
%  Add 'draft' option to mark overfull boxes with black boxes
%  Add 'showpacs' option to make PACS codes appear
%  Add 'showkeys' option to make keywords appear
%\documentclass[aps,prl,preprint,groupedaddress]{revtex4-1}
%\documentclass[aps,prl,preprint,superscriptaddress]{revtex4-1}
%\documentclass[aps,prl,reprint,groupedaddress]{revtex4-1}

\documentclass[%
 reprint,
%superscriptaddress,
%groupedaddress,
%unsortedaddress,
%runinaddress,
%frontmatterverbose,
%preprint,
%showpacs,preprintnumbers,
%nofootinbib,
%nobibnotes,
%bibnotes,
 amsmath,amssymb,
 aps,
%pra,
%prb,
%rmp,
%prstab,
%prstper,
%floatfix,
]{revtex4-1}


\usepackage{graphicx}
\usepackage[space]{grffile}
\usepackage{latexsym}
\usepackage{textcomp}
\usepackage{longtable}
\usepackage{tabulary}
\usepackage{booktabs,array,multirow}
\usepackage{amsfonts,amsmath,amssymb}
\usepackage{natbib}
\usepackage{url}
\usepackage{hyperref}
\hypersetup{colorlinks=false,pdfborder={0 0 0}}
\usepackage{etoolbox}
\makeatletter
\patchcmd\@combinedblfloats{\box\@outputbox}{\unvbox\@outputbox}{}{%
  \errmessage{\noexpand\@combinedblfloats could not be patched}%
}%
\makeatother
% You can conditionalize code for latexml or normal latex using this.
\newif\iflatexml\latexmlfalse
\providecommand{\tightlist}{\setlength{\itemsep}{0pt}\setlength{\parskip}{0pt}}%

\AtBeginDocument{\DeclareGraphicsExtensions{.pdf,.PDF,.eps,.EPS,.png,.PNG,.tif,.TIF,.jpg,.JPG,.jpeg,.JPEG}}

\usepackage[utf8]{inputenc}
\usepackage[ngerman,english]{babel}


%\usepackage{babel}



\begin{document}

\title{Can we run quantum circuits on ultra-cold atom devices?}

\author{Fred Jendrzejewski}
\affiliation{Kirchhoff-Institut für Physik}

\author{Manuel Rudolph}
\affiliation{Kirchhoff-Institut für Physik}




\selectlanguage{english}
\begin{abstract}
In this blog-post, we present our path and thoughts towards using
ultra-cold atom experiments for quantum computation. They are the result
of a two month internship where we studied the feasibility of such an
undertaking in our group. Many associate only universal devices,
especially qubit devices, to be valid quantum computers. We show how we
think of our ultra-cold atoms in terms of quantum circuits and implement
first steps in the software
framework~\href{https://pennylane.ai/}{PennyLane}.%
\end{abstract}%



%\pacs{Valid PACS appear here}% PACS, the Physics and Astronomy
                             % Classification Scheme.
%\keywords{Suggested keywords}%Use showkeys class option if keyword
                              %display desired
\maketitle


\section*{Motivation}

{\label{773840}}

Over recent years, interest in quantum information processing has
increased tremendously. This rise was fueled by activities of massive
commercial players
like~{\href{https://www.ibm.com/quantum-computing/}{IBM},}
\href{https://research.google/teams/applied-science/quantum/}{Google},
\href{https://www.microsoft.com/en-us/quantum}{Microsoft} and large
investments in startups like \href{https://www.dwavesys.com/}{D-Wave},
\href{https://rigetti.com/}{Rigetti},
\href{https://www.xanadu.ai/}{Xanadu},
\href{https://www.zapatacomputing.com/}{Zapata}and
\href{https://cambridgequantum.com/}{Cambridge Quantum Computing}. The
work of these companies typically focuses on \textbf{universal quantum
computers}, which hold the premise of exponential speed-up for a variety
of NP-hard problems. However, this \textbf{universality usually comes at
the price} of algorithmic overhead for a wide range of tasks which
generally leads to worse overall fidelity. If we give up the constraint
of a universal quantum computer, we can build \textbf{specialized
quantum hardware} for the problems of interest. Such systems are
called~\textbf{quantum simulators}.~

\textbf{Ultra-cold atoms~have become}~\textbf{a leading platform of such
quantum simulators at a large scale}. Most importantly for us, they are
the experimental platform our research group is working on. Already
demonstrated applications involve an enormous variety of
condensed-matter problems like e.g. the Hubbard model
\cite{Bloch_2008}, topological systems \cite{Goldman_2014} ,
superfluidity \cite{Regal_2007} and disorder \cite{Lagendijk_2009} . More
recently, cold atoms started to find applications to Ising models
\cite{Bernien2017} and high-energy physics \cite{Zohar2016}. ~

However, \textbf{quantum simulators
are}~\textbf{currently}~\textbf{limited to an academic environment}
because of~

\begin{enumerate}
\tightlist
\item
  the \textbf{complexity of constructing the experimental hardware},
  which involves substantial technical know-how and funding.
\item
  the \textbf{specificity of knowledge and language required} to
  understand/communicate the experiment.
\item
  the need to \textbf{formulate the target problem as a Hamiltonian}.
\item
  the~\textbf{manual}~\textbf{compilation of programs} on this hardware
  class.~
\end{enumerate}

A~\textbf{circuit-based approach to ultra-cold atom quantum simulators}
would alleviate points 2) and 3) as we could agree on the quantum
circuit, given the hardware-specific operations, and run it on the
experiment. Integrating the system with a software stack
like~\href{https://pennylane.ai/}{PennyLane~}would additionally allow to
address and run the experiment on a higher level
and~\textbf{substantially lower the barrier of entry for users outside
of the cold-atom and quantum physics community}. ~

These points motivated our research group to have a look how our
experimental hardware, which is controlled by
the~\href{http://labscriptsuite.org/}{Labscript Suite},~can be
integrated into PennyLane to run quantum circuits on ultra-cold atom
devices.~ Our group is part of
\href{http://www.kip.uni-heidelberg.de/synqs/}{SynQS}, working on two
ultra-cold atom experiments, the NaLi and SoPa (named after the two
atomic species used in the lab).

\par\null

\subsection*{Why use ultra-cold atoms as a quantum information platform
?}

{\label{677154}}

For ultra-cold atoms, the~\textbf{processing unit is not a qubit} and as
such, the \textbf{natural operations are not} generally
\textbf{the~commonly used and widely known qubit operations} such
as~\(X,Y,Z\) rotations, Hadamard~\(H\) gates,
entangling~\(CNOT \) or~\(XX\) gates with
oberservables like~\(\sigma^z\) expectations.~{{Instead, our NaLi
experiment implements a~}}\(X(\theta)\){ \textbf{rotation on a long
spin} of many bosons on one optical lattice site. Additionally, the
system naturally evolves under a many-body
Hamiltonian~}\(H_{mb}\), coupling the the atomic species.

What is the qubit equivalent of those operations?~{The relation is
pretty complex -} and that is the nice thing!~\textbf{Cold atoms are
different}. Given their success in the quantum simulation of many-body
problems, they could make for an exciting quantum information processor
that is~\textbf{complementary to~}{\textbf{universal devices.}} So as a
general rule of thumb, it could be argued that cold atom machines often
give up some control over individual particles, leading to much bigger
systems. Additionally, because we work with a large number of atoms, we
can measure obtain rather precise estimates of expectation
values~\textbf{{{with a~single state preparation and one
measurement}}}.~

\par\null

\section*{A concrete example from our
group}

{\label{216843}}

Before we go into more detailed discussion of our implementation, we
will present a \textbf{concrete example of our device}, where
\textbf{we} \textbf{study lattice gauge theories}. These theories have
become a popular benchmark for quantum simulators~\cite{Kokail_2019} and
quantum computers~\cite{Martinez_2016,tavernelli2020} in recent years. On our NaLi
machine, we recently performed experiments on the building block for
certain quantum simulators that would be suited for theories from
high-energy physics~\cite{Mil2020}. ~These experiments are performed
with two atomic species, sodium and lithium, that can be prepared in two
internal spin states.~

\textbf{Re-thinking our analog quantum simulation} for the purpose of
quantum circuits, we can \textbf{represent the sodium atoms as a very
long spin} state~\(\left|\psi_N\right\rangle\). The observable outcomes
are~\(L_z=-\frac{N}{2},\cdots,\frac{N}{2}\), where~\(N\sim10^5\) is the number of sodium
atoms. On a quantum computer this should be compared to a qubit, for
which we can only measure two outcomes~\(L_z=\pm\frac{1}{2}\).

The \textbf{lithium atoms} on the other hand can be \textbf{described by
the states on two independent sites~}\(\left|\psi_p\right\rangle\)
and~\(|\psi_v\rangle\). For each site we can observe that number of
atoms that sit on it. The observable outcomes are
then~\(n_{p/v} = 0, \dots, n\) where~\(n\sim 10^4\) is the number of
lithium atoms.~

\par\null

In this formulation and \textbf{analogous to other circuit-based quantum
computing devices}, the NaLi experiment then consists of three main
stages:

\begin{enumerate}
\tightlist
\item
  The atoms are prepared in some initial state\ldots{}
\item
  controlled through a set of operations/gates\ldots{}
\item
  and then measured to evaluate the operators~\(L_z\)
  and~\(N_{p,v}\)
\end{enumerate}

Without having to dig through all the physical details to understand the
publication~\cite{Mil2020}, the figures in it intuitively open
themselves up to \textbf{readers with varying backgrounds} when plotting
the \textbf{corresponding circuits} next to them as shown in Fig.
{\ref{428832}}.\selectlanguage{english}
\begin{figure}[h!]
\begin{center}
\includegraphics[width=0.70\columnwidth]{figures/AlexAsCircuit/mil-with-circuits}
\caption{{Scetched quantum circuits that correspond to selected figures from Mil
et al.~\protect\cite{Mil2020}. The results gain a clear intuitive meaning
through the quantum circuits even though the system that is being
simulated is not known. The color blue signifies the Sodium atoms and
the orange color Lithium.
{\label{428832}}%
}}
\end{center}
\end{figure}

\par\null

\textbf{Additional benefits of using the circuit language} include

\begin{itemize}
\tightlist
\item
  a \textbf{more uniform definition of~}\emph{\textbf{fidelity}} for
  ultra-cold atom experiments relative to fidelity in qubit devices
  where each operation and the final result can be quantified.
\item
  \textbf{improved communication} with \textbf{and feedback} of
  theorists involved in the project by working on the same level of
  resolution.
\item
  more easily \textbf{generalizing the capabilities of the device} to
  work on \textbf{other problems.}
\item
  general accessibility and a~\textbf{path towards using ultra-cold
  atoms experiments in the quantum computing community.}
\end{itemize}

\section*{}\label{section}

\section*{Choice of framework software for our quantum
circuits}\label{choice-of-framework-software-for-our-quantum-circuits}

The choice of which framework software to use for our circuit-based
approach was not a particularly hard one for us. We have our own
experiment in the lab which implements a non-standard device for quantum
simulation and thus some of our requirements are (roughly in order of
importance):

\begin{enumerate}
\tightlist
\item
  Can we \textbf{integrate our own experimental hardware} into the
  framework and potentially run it from there?
\item
  How strong of an \textbf{assumption} is made \textbf{about the
  computational unit} of the hardware? Does it have to be strictly
  qubits/photons etc.?
\item
  Is the framework \textbf{open-source} and shows \textbf{openness to
  the academic community}?
\item
  Can also \textbf{integrate classical simulators} that are based on our
  hardware?
\item
  Does it provide some sort of \textbf{user management and safety
  measures} to avoid unqualified users from killing the system?
\end{enumerate}

Lets give a brief overview of how we answered those questions with
regard to PennyLane.~\textbf{PennyLane is designed for users to write
and use their own~}\emph{\textbf{plugin}}, which is what they call the
integration of quantum computing devices into the framework. Officially,
it only supports~\emph{qubit~}and~\emph{continuous variable} devices and
not more general approaches but we may be able to~\textbf{integrate
bosonic and fermionic atoms flexibly} as the code itself is not very
restricted. Other big platorms
like~\href{https://qiskit.org/}{Qiskit},~\href{http://docs.rigetti.com/en/stable/}{Forest}
and~\href{https://strawberryfields.readthedocs.io/en/stable/}{Strawberry
Fields} focus exclusively on their own quantum computing hardware or
simulator without the possibility to integrate your own. PennyLane
is~\textbf{open-source and Python-based,} which is perfect as the
\emph{Labscript Suite} controlling our experiment is so, too.
Additionally, the company behind
PennyLane,~\href{https://www.xanadu.ai/}{Xanadu}, is a partly academic
player who is very open in their publications and \textbf{supports a
great community} around the software. The framework allows to add
simulators to the framework device which mirror the functionality of the
hardware. It performs various parameter range checks for the system and
the operations used. Not built-in is user a system for management which
allows only registered users to run the experiment from a queue. We
would have to implement this ourselves.

All-in-all,~\textbf{PennyLane seems to be the best and likely only fit}
for our endeavors. Let's talk about it in more detail.

\section*{Pennylane}

{\label{935975}}

\begin{quote}
\emph{A cross-platform Python library for quantum machine learning,
automatic differentiation, and optimization of hybrid quantum-classical
computations.}
\end{quote}\selectlanguage{english}
\begin{figure}[h!]
\begin{center}
\includegraphics[width=0.63\columnwidth]{figures/PennyLane-puzzle/PennyLane-puzzle}
\caption{{PennyLane as the connecting puzzle piece.
\href{https://pennylane.readthedocs.io/en/stable/introduction/pennylane.html}{Source}
{\label{477211}}%
}}
\end{center}
\end{figure}

\href{https://pennylane.ai/}{PennyLane}\protect\hyperlink{killoran2018}{(Bergholm
2018)} is created by
\emph{\href{https://www.xanadu.ai/}{Xanadu~}}and~presents itself as
the~\textbf{connecting puzzle piece} \textbf{between powerful
established python libraries for machine learning, and quantum computing
platforms}. It provides readily implemented quantum algorithms for
quantum machine learning, quantum chemistry and optimization. A
particular algorithm provided in PennyLane is automatic differentiation
of quantum circuits which allows to perform shot-efficient application
of classical-quantum algorithms such as the~\emph{Variational Quantum
Eigensolver} (VQE). ~~

Existing popular quantum computing platforms such as
\href{https://qiskit.org/}{Qiskit}for qubits and
\href{https://strawberryfields.readthedocs.io/en/stable/\#}{Strawberry
Fields} (also created by \emph{Xanadu}) for photons have written plugins
for PennyLane which make their \textbf{quantum simulators or quantum
hardware accessible through it} and with that PennyLane's quantum
algorithms and automatic differentiation feature.~

In fact,~\textbf{any quantum lab can adapt
the~}\href{https://github.com/XanaduAI/pennylane-plugin-template}{\textbf{plugin
template}} \textbf{for their own experiment and simulators} to
assimilate their own specific workflow with PennyLane where users are
generally able to use the same workflow accross platforms and know what
to expect. Fig.~{\ref{757073}} shows a typical way of
using PennyLane in a classical-quantum hybrid algorithm. Here, the
default qubit simulator by PennyLane is used. After writing our own
plugin, we can call that device, change the quantum circuit to contain
the supported operations and the workflow is otherwise the same.
All-in-all, \textbf{the interface is very robust towards changing up the
quantum backend} (especially when they support the same operations).\selectlanguage{english}
\begin{figure}[h!]
\begin{center}
\includegraphics[width=0.35\columnwidth]{figures/PennyLane-use/PennyLane-use}
\caption{{Typical simple way of using PennyLane by instantiating the device,
defining the parametrized quantum circuit and the cost, and calculating
the gradient for the classical parameters.
\href{https://pennylane.readthedocs.io/en/stable/introduction/pennylane.html}{Source}
{\label{757073}}%
}}
\end{center}
\end{figure}

\section*{}\label{section}

\section*{Our workflow and connecting PennyLane to our
hardware}\label{our-workflow-and-connecting-pennylane-to-our-hardware}

The current state of our \textbf{\emph{pennylane\_ls}}~\textbf{plugin
for ultra-cold atom experiments} can be viewed
in~\href{https://github.com/synqs/pennylane-ls}{\textbf{this public
github repository}}. It provides examples from our labs and can guide
you to adapting it to your own experiment and needs.

Our group uses \textbf{} the
\href{http://labscriptsuite.org/}{\textbf{Labscript suite}} \textbf{to
coordinate and execute experimental sequences} with cold atoms. ~

\begin{quote}
~\emph{The labscript suite is a collection of programs, which work
together to form a control system for autonomous, hardware timed
experiments}. ~
\end{quote}

On the deepest level it is very nuts and bolts with plenty of commands
in the style of Fig. {\ref{799186}} below.\selectlanguage{english}
\begin{figure}[h!]
\begin{center}
\includegraphics[width=0.98\columnwidth]{figures/Labscript-NaLi-code/Labscript-NaLi-code}
\caption{{The Experiment.py (left) defines the experimental sequence of a
Labscript-controlled lab experiment, here by calling functions from
NaLiFunctions.py at given times. The function UMPUMP (right) shows the
exact low level hardware instructions that are to be executed.
{\label{799186}}%
}}
\end{center}
\end{figure}

The~\emph{\textbf{Experiment.py}} \textbf{file~}(Fig.
{\ref{799186}}) \textbf{determines the exact sequence
of hardware controls} that is executed during the measurement. It
contains Labscript syntax and is fed into the Labscript Python
compiler~\emph{\textbf{Runmanager}.~~}

Our approach for \textbf{combining Labscript with PennyLane} is
sumarized in Fig.~{\ref{547624}}.

With the help of the plugin functions~\emph{\textbf{pre\_apply()},
\textbf{apply()}~}and~\emph{\textbf{post\_apply()}}, \textbf{we generate
a new~}\emph{\textbf{Experiment.py}} \textbf{on the fly} depending on
the quantum circuit programmed through the PennyLane interface.\selectlanguage{english}
\begin{figure}[h!]
\begin{center}
\includegraphics[width=0.56\columnwidth]{figures/pre-post-apply/pre-post-apply}
\caption{{The functions~\emph{pre\_apply(), apply()~}and~\emph{post\_apply()} are
used to generate a new~\emph{Experiment.py} on the fly for each quantum
circuit in PennyLane.~\emph{apply()} loops through the operations in the
circuit while the other functions always write the same necessary code
for the compilation.
{\label{547624}}%
}}
\end{center}
\end{figure}

In other words, \textbf{our PennyLane plugin translates a quantum
circuit to an} \emph{\textbf{Experiment.py}} which our control system
can interpret.

The~\emph{\textbf{expval()}}~method from the plugin is then used to
\textbf{send the newly generated file to the~}\emph{\textbf{Runmanager}}
and engage the sequence. The output shots are then evaluated and the
method returns an expectation.

A detailled discussion of the technical details can be found in the
README.md of the~\href{https://github.com/synqs/pennylane_ls}{public
repository}. ~The \textbf{features of our plugin} currently include:

\begin{itemize}
\tightlist
\item
  \textbf{Proposing a higher-level control of experiments} that are run
  controlled by the Labscript Suite through PennyLane.
\item
  A \textbf{framework device that provides a template for labs with
  Labscript} to quickly adapt code for their own setup.
\item
  Including \textbf{two devices of the SynQS group at the
  Kirchhoff-Institute} \textbf{for Physics} of Prof. Fred Jendrzejewski.
\end{itemize}

The framework device in~\emph{SynQSDevice.py} is written such that
\textbf{a group which uses the Labscript Suite} to run their experiment
\textbf{can} \textbf{adapt the module easily to their own setup}.

\par\null

Finishing up this project,~we have \textbf{succeed in providing first
circuit control to our experiments~}through PennyLane and extract the
results. However, much \textbf{further automation will be needed} to
fully leverage the possibilities of the quantum circuits that we can run
on our ultra-cold atom devices.

\section*{Future steps}\label{future-steps}

To get the project \textbf{towards} \textbf{a full Pennylane plugin}, we
have to:

\begin{enumerate}
\tightlist
\item
  Give \emph{\textbf{wires}} \textbf{} a \textbf{} meaning.
\item
  \textbf{Operations need to be restricted to certain wires}, especially
  when different atomic species are involved.
\item
  Implement~\textbf{true remote control} of the lab through the
  \emph{pennylane\_ls} module.
\item
  Implement \textbf{full result post-processing} that is currently used
  in the lab.
\item
  Implement some kind of \textbf{user management}.
\end{enumerate}

While it seems quite clear how we can implement points 1 - 4, it seems
to be a whole new problem to implement some kind of user management for
our systems.~

In any case the \textbf{first steps are remarkably promising} and we
will see how the quantum circuits can be efficiently employed in
ultra-cold atom experiments.

\selectlanguage{english}
\FloatBarrier
\nocite{*}

\bibliographystyle{apsrev4-1}
\bibliography{bibliography/converted_to_latex.bib%
}

\end{document}

